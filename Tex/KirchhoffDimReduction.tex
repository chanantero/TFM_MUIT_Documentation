%\chapter{Dimensionality reduction: Kirchhoff-Helmholtz 2.5D Integral}
For practical reasons, it is convenient to reduce the surface $\surfaceTheo$ to a closed line $\sectionTheo$ contained on a plane, that would correctly synthesize the field on that plane, as it is proposed in \cite{Vogel} and developed in \cite{Start1997} and, only for the Rayleigh integrals, in \cite{Verheijen}. The way to do it is the next.

Let's assume that all primary sources and the positions where we want to synthesize their field are all located in the same plane $\wfsPlane$. We also assume that $\surfaceTheo$ is a surface formed by all the lines that are parallel to the normal vector of $\wfsPlane$, $\wfsPlaneNormal$, and intersect with a closed 2D curved $\sectionTheo$ contained on $\wfsPlane$. In other words, it is like a tube infinitely long that extends in the $\wfsPlaneNormal$ direction and it's section (the intersection of $\surfaceTheo$ and the plane $\wfsPlane$) is $\sectionTheo$, as shows \autoref{figCylindricalSurface}.
\showto{private}{Other ways of expressing it is:
	\begin{itemize}
		\item the projection of $\surfaceTheo$ onto $\wfsPlane$ is equal to their intersection $\sectionTheo$
		\item $\surfaceTheo$ can be built by the extension of $\sectionTheo$ along the normal direction to the plane ($\wfsPlaneNormal$)
		\item at each point of $\sectionTheo$, a vertical (vertical is normal to the plane) line is extended
		\item it consists on all the lines parallel to the y axis that intersect with $\sectionTheo$
		\item $\surfaceTheo$ is a cylindrical surface whose generating curve is $\sectionTheo$ and the rulings are parallel to the y axis
		\item A given point of $\surfaceTheo$ (let's call it $\PosTheo[surface]$) is the result of adding a height $h$ to a point of $\sectionTheo$ ($\PosTheo[section]$), which is actually the projection of $\PosTheo[surface]$ onto $\wfsPlane$ ($\PosTheo[section] = \vectorproj[\wfsPlaneNormal]{\PosTheo[surface]}$): $\PosTheo[surface] = \PosTheo[section] + h\wfsPlaneNormal$
	\end{itemize}}

\begin{figure}[h]
	\centering
	\def\svgwidth{1\columnwidth}
	\graphicspath{{../TFM/Img/}}
	\input{../TFM/Img/Kirchoff2_5DTheoScheme.pdf_tex}
	\caption{Scheme of Kirchhoff integral dimensionality reduction}
\end{figure}

In order to perform the dimensionality reduction, we calculate the contribution of each of lines that form $\surfaceTheo$, one for each of the points $\PosTheo[section]$ that constitute the curve $\sectionTheo$.

\begin{gather}
	\Field(\PosTheo) = \int_{\sectionTheo} \left( I_{1}(\PosTheo[section], \PosTheo) + I_2(\PosTheo[section], \PosTheo) \right)
	d\PosTheo[section] \\
	I_{1}(\PosTheo[section], \PosTheo) = \frac{-1}{4\pi}  \int_{-\infty}^{\infty} G(\PosTheo[section] + h\wfsPlaneNormal  \vert \PosTheo) \frac{\partial P(\PosTheo[section] + h\wfsPlaneNormal)}{\partial \mathbf{n}} dh \label{lineIntI} \\	I_{2}(\PosTheo[section], \PosTheo) = \frac{1}{4\pi} \int_{-\infty}^{\infty} P(\PosTheo[section] + h\wfsPlaneNormal)\frac{\partial G(\PosTheo[section] + h\wfsPlaneNormal\vert \PosTheo)}{\partial \mathbf{n}} dh \label{lineIntII}
\end{gather}

The field $P$ is assumed to be produced by a punctual source:
\begin{equation}
\Field(\PosTheo) = \Field[primarySource](\PosTheo) = \CoefTheo[primarySource] \Diag\left(\normalized{\PosTheo - \PosTheo[primarySource]}\right) \GreenFunc[{\PosTheo - \PosTheo[primarySource]}]
\end{equation}
where $\CoefTheo$ is the signal strength of the source (complex scalar) and $\Diag\left(\normalized{\PosTheo - \PosTheo[primarySource]}\right)$ is the directivity pattern ofthe source in the $\normalized{\PosTheo - \PosTheo[primarySource]}$ direction.

\begin{shownto}{private}
Without loss of generalization, let's assume that $\wfsPlane$ is the $XZ$ plane ($y = 0$). This means that $\PosTheo = [\PosTheo[noValue][x], 0, \PosTheo[noValue][z]]$, $\PosTheo[primarySource] = [\PosTheo[primarySource][x], 0, \PosTheo[primarySource][z]]$ and $\wfsPlaneNormal = \ydir$. The directivity can be expressed as $\Diag\left(\azimuth, \elevation\right)$, where $\left(\azimuth, \elevation\right)$ expresses the direction $\normalized{\PosTheo - \PosTheo[primarySource]}$ in a spherical coordinate system:
\begin{gather}
	\frac{\PosTheo - \PosTheo[primarySource]}{\norm{\PosTheo - \PosTheo[primarySource]}} = [\cos{\elevation} \sin\azimuth, \sin\elevation, \cos\elevation \cos\azimuth]^T \\
	\azimuth = \arctan \frac{\PosTheo[noValue][x] - \PosTheo[primarySource][x]}{\PosTheo[noValue][z] - \PosTheo[primarySource][z]} \qquad \elevation = \arctan{\frac{\PosTheo[noValue][y] - \PosTheo[primarySource][y]}{\sqrt{(\PosTheo[noValue][x] - \PosTheo[primarySource][x])^2 + (\PosTheo[noValue][z] - \PosTheo[primarySource][z])^2}}} 
\end{gather}
\end{shownto}

\begin{shownto}{public}
The approximated solution provided by the stationary phase method is (\cite[Equation 3.17]{Start1997}):
\end{shownto}
\begin{shownto}{private}
Then,
\begin{equation}
\frac{\partial P(\PosTheo)}{\partial \mathbf{n}} = \langle \nabla P(\PosTheo), \vec{n} \rangle
\end{equation}
\begin{multline}
\nabla P(\PosTheo) = \frac{\partial P(\PosTheo)}{\partial x} \hat{\vec{x}} + \frac{\partial P(\PosTheo)}{\partial y} \hat{\vec{y}} + \frac{\partial P(\PosTheo)}{\partial z} \hat{\vec{z}} =
\left\{ \begin{matrix}
\nabla \azimuth = \frac{\cos\azimuth}{\norm{\PosTheo - \PosTheo[primarySource]} \cos \elevation} \hat{\vec{x}} +
\frac{-\sin \azimuth}{\norm{\PosTheo - \PosTheo[primarySource]} \cos \elevation} \hat{\vec{z}} \\
\nabla \elevation = \frac{-\sin \elevation \sin \azimuth}{\norm{\PosTheo - \PosTheo[primarySource]}} \hat{\vec{x}} + \frac{\cos \elevation}{\norm{\PosTheo - \PosTheo[primarySource]}} \hat{\vec{y}} + \frac{-\sin \elevation \cos \azimuth}{\norm{\PosTheo - \PosTheo[primarySource]}} \hat{\vec{z}}
\end{matrix} \right\}
= S(\omega) \GreenFunc[{\PosTheo - \PosTheo[primarySource]}]
\\ \left(D(\azimuth, \elevation, \omega)
\left(-jk - \frac{1}{\norm{\PosTheo - \PosTheo[primarySource]}}\right) \frac{\PosTheo - \PosTheo[primarySource]}{\norm{\PosTheo - \PosTheo[primarySource]}} +
\frac{\partial D}{\partial \azimuth} \frac{\cos \azimuth \hat{\vec{x}} - \sin \azimuth \hat{\vec{z}}}{\norm{\PosTheo - \PosTheo[primarySource]} \cos \elevation} + \frac{\partial D}{\partial \elevation} \frac{-\sin\elevation \sin\azimuth \hat{\vec{x}} + \cos\elevation \hat{\vec{y}} - \sin\elevation \cos\azimuth \hat{\vec{z}}}{\norm{\PosTheo - \PosTheo[primarySource]}} \right)
\end{multline}

Along a given vertical line ($\PosTheo[section] + y$), the unitary inward normal pointing vector is:
\begin{equation}
\vec{n} = [n_x, 0, n_z]^T = [\sin\azimuthNormal, 0, \cos\azimuthNormal]
\end{equation}
which remains constant along that line and its component in $\hat{\vec{y}} = 0$.

Hence,
\begin{multline}
\frac{\partial P(\PosTheo[surface])}{\partial \mathbf{n}} = \langle \nabla P(\PosTheo[surface]), \vec{n} \rangle = \left\{
\begin{matrix}
	\cos\alpha \sin\beta - \sin\alpha \sin\beta = sin(\beta - \alpha) \\ \cos\alpha \cos\beta + \sin\alpha\sin\beta = \cos(\beta - \alpha)
\end{matrix}  \right\}
= S(\omega) \GreenFunc[{\PosTheo[surface] - \PosTheo[primarySource]}] \\ \left[ D(\azimuth, \elevation, \omega) \left(-jk - \frac{1}{\norm{\PosTheo[surface] - \PosTheo[primarySource]}}\right) \cos\elevation \cos\normPrimaryPropAngleSection + \frac{\partial D}{\partial \azimuth} \frac{\sin \normPrimaryPropAngleSection}{\norm{\PosTheo[surface] - \PosTheo[primarySource]} \cos \elevation} + \frac{\partial D}{\partial \elevation} \frac{-\sin\elevation \cos\normPrimaryPropAngleSection}{\norm{\PosTheo[surface] - \PosTheo[primarySource]}}  \right]
\label{directDerivField}
\end{multline}
where $\normPrimaryPropAngleSection = \azimuthNormal - \azimuth$.

According to the stationary phase point method,
\begin{equation}
	I = \int_{-\infty}^{+\infty} f(y) e^{j\phi(y)} dy \approx f(y_0) e^{j\phi(y_0)} \sqrt{\frac{j 2\pi}{\phi''(y_0)}}
\end{equation}
where $y_0$ is the value of $y$ where the phase gets stationary $\frac{d\phi(y)}{dy} = 0$, and $\phi''(y_0)$ is the second derivative of $\phi$ evaluated at $y_0$. Intuitively it means that the integral of a phase changing function only has a significant contribution in the region where the phase change slows down.

Substituting \autoref{directDerivField} in \autoref{lineIntI}, and taking in account that $\PosTheo[surface] = \PosTheo[section] + y\hat{\vec{y}}$ and applying the stationary phase method:
\begin{align}
f(y) &= \frac{-S(\omega)}{4\pi \norm{\PosTheo[surface] - \PosTheo[primarySource]} \norm{\PosTheo[surface] - \PosTheo}} \\ 
&\left[ D(\azimuth, \elevation, \omega) \left(-jk - \frac{1}{\norm{\PosTheo[surface] - \PosTheo[primarySource]}}\right) \cos\elevation \cos\normPrimaryPropAngleSection + \frac{\partial D}{\partial \azimuth} \frac{\sin\normPrimaryPropAngleSection}{\norm{\PosTheo[surface] - \PosTheo[primarySource]} \cos \elevation} + \frac{\partial D}{\partial \elevation} \frac{-\sin\elevation \cos\normPrimaryPropAngleSection}{\norm{\PosTheo[surface] - \PosTheo[primarySource]}}  \right] \nonumber \\
\phi(y) &= -k\left(\norm{\PosTheo[surface] - \PosTheo[primarySource]} + \norm{\PosTheo[surface] - \PosTheo}\right)\\
y_0 &= 0 \\
\phi(y_0) &= -k(\distLinePrimSource + \distLinePoint) \\
f(y_0) &= \left\{ \elevation(y_0) = 0 \right\} = \frac{-S(\omega)}{\distLinePrimSource \distLinePoint} \left[ D(\azimuth, 0, \omega) \left(-jk - \frac{1}{\distLinePrimSource}\right) \cos\normPrimaryPropAngleSection + \frac{\partial D}{\partial \azimuth} \frac{\sin\normPrimaryPropAngleSection}{\distLinePrimSource} \right] \\
\phi''(y_0) &= \frac{d^2\phi(y)}{dy^2}\rvert_{y = y_0} = -k\frac{\distLinePrimSource + \distLinePoint}{\distLinePrimSource \distLinePoint} \\
I &\approx \frac{-S(\omega)}{4\pi \distLinePrimSource \distLinePoint} \left[ D(\azimuth, 0, \omega) \left(-jk - \frac{1}{\distLinePrimSource}\right) \cos\normPrimaryPropAngleSection + \frac{\partial D}{\partial \azimuth} \frac{\sin\normPrimaryPropAngleSection}{\distLinePrimSource} \right] e^{-jk(\distLinePrimSource + \distLinePoint)} \sqrt{\frac{j 2 \pi}{-k\frac{\distLinePrimSource + \distLinePoint}{\distLinePrimSource \distLinePoint}}}
\end{align}

where $\distLinePrimSource = \norm{\PosTheo[section] - \PosTheo[primarySource]} = \sqrt{z_{\PosTheoSubInd[primarySource]}^2 + (x_{\PosTheoSubInd[surface]} - x_{\PosTheoSubInd[primarySource]})^2}$ and $\distLinePoint = \norm{\PosTheo[section] - \PosTheo} = \sqrt{z^2 + (x_{\PosTheoSubInd[surface]} - x)^2}$.

Finally, assuming $k\distLinePrimSource >> 1$ and changing $D(\azimuth, 0, \omega)$ by its equivalent expression in Cartesian coordinates $\Diag\left(\normalized{\PosTheo[section] - \PosTheo[primarySource]}\right)$:
\end{shownto}
\begin{equation}
I_1(\PosTheo[section], \PosTheo) \approx \widetilde{I_1}(\PosTheo[section], \PosTheo) = \frac{1}{2} \CoefTheo[primarySource] \Diag\left(\normalized{\PosTheo[section] - \PosTheo[primarySource]}\right) \cos\normPrimaryPropAngleSection \frac{e^{-jk\distLinePrimSource}}{\sqrt{\distLinePrimSource}} \frac{e^{-jk\distLinePoint}}{\distLinePoint} \sqrt{\frac{jk}{2\pi}} \sqrt{\frac{\distLinePoint}{\distLinePrimSource + \distLinePoint}}
\label{I1approx}
\end{equation}
\showto{public}{where $\distLinePrimSource = \norm{\PosTheo[section] - \PosTheo[primarySource]}$ is the distance between the primary source and the secondary source and $\distLinePoint = \norm{\PosTheo - \PosTheo[section]}$ is the distance between the secondary source and the receiving point, and $\normPrimaryPropAngleSection$ is the angle between the inward normal vector $\surfaceNormal$ and the propagation direction from the primary to the source $\PosTheo[section] - \PosTheo[primarySource]$. The result is valid for $k\distLinePrimSource >> 1$.
}

Let's see that in previous equation there is the expression for the propagation of a monopole. So, we can express it as the strength of a secondary monopole source $\CoefTheo[section][monopole][kirchhoff](\PosTheo[section], \PosTheo)$ propagated to the receiving point $\PosTheo$:
\begin{equation}
\begin{aligned}
\widetilde{I_1}(\PosTheo[section], \PosTheo) &= \CoefTheo[section][monopole][kirchhoff](\PosTheo[section], \PosTheo) \frac{e^{-jk\distLinePoint}}{\distLinePoint} \\
\CoefTheo[section][monopole][kirchhoff](\PosTheo[section], \PosTheo) &= \frac{1}{2}\CoefTheo[primarySource] \Diag\left(\normalized{\PosTheo[section] - \PosTheo[primarySource]}\right) \cos\normPrimaryPropAngleSection \frac{e^{-jk\distLinePrimSource}}{\sqrt{\distLinePrimSource}}\sqrt{\frac{jk}{2\pi}} \sqrt{\frac{\distLinePoint}{\distLinePrimSource + \distLinePoint}}
\end{aligned}
\end{equation}

The same reasoning can be applied to \autoref{lineIntII}, and the result is \cite[Equation 3.24]{Start1997}:
\begin{equation}
I_2(\PosTheo[section]) \approx \widetilde{I_2}(\PosTheo[section]) = \frac{1}{2} \CoefTheo[primarySource] \Diag\left(\normalized{\PosTheo[section] - \PosTheo[primarySource]}\right) \cos\normSecondPropAngleSection \frac{e^{-jk\distLinePrimSource}}{\sqrt{\distLinePrimSource}} \frac{e^{-jk\distLinePoint}}{\distLinePoint} \sqrt{\frac{jk}{2\pi}} \sqrt{\frac{\distLinePoint}{\distLinePrimSource + \distLinePoint}}
\label{I2approx}
\end{equation}
where $\normSecondPropAngleSection$ is the angle between the inward normal vector $\surfaceNormal$ and the propagation direction from the secondary source to the receiving point $\PosTheo - \PosTheo[section]$. Hence, $\cos\normSecondPropAngleSection = \scalarProd{\normalized{\PosTheo - \PosTheo[section]} }{\surfaceNormal}$.

As with \autoref{I1approx}, \autoref{I2approx} can also be expressed as the strength of a secondary source $\CoefTheo[section][dipole][kirchhoff](\PosTheo[section], \PosTheo)$ propagated to the receiving point, but this time the secondary source is a dipole with the inward normal vector as broadside direction:
\begin{equation}
\begin{aligned}
\widetilde{I_2}(\PosTheo[section]) &= \CoefTheo[section][dipole][kirchhoff](\PosTheo[section], \PosTheo) \frac{e^{-jk\distLinePoint}}{\distLinePoint} \cos\normSecondPropAngleSection \\
\CoefTheo[section][dipole][kirchhoff](\PosTheo[section], \PosTheo) &= \frac{1}{2} \CoefTheo[primarySource] \Diag\left(\normalized{\PosTheo[section] - \PosTheo[primarySource]}\right) \frac{e^{-jk\distLinePrimSource}}{\sqrt{\distLinePrimSource}} \sqrt{\frac{jk}{2\pi}} \sqrt{\frac{\distLinePoint}{\distLinePrimSource + \distLinePoint}}
\end{aligned}
\end{equation}

The synthesized field will be then:
\begin{multline}
	\Field[kirchhoff](\PosTheo) = \int_{S_0} \left( \widetilde{I_{1}}(\PosTheo[section], \PosTheo) + \widetilde{I_2}(\PosTheo[section], \PosTheo) \right)
	\dif[\PosTheo[section]] = \\ \int_{S_0} \left( \CoefTheo[section][monopole][kirchhoff](\PosTheo[section], \PosTheo) \frac{e^{-jk\distLinePoint}}{\distLinePoint} + \CoefTheo[section][dipole][kirchhoff](\PosTheo[section], \PosTheo) \frac{e^{-jk\distLinePoint}}{\distLinePoint} \cos\normSecondPropAngleSection \right)
	\dif[\PosTheo[section]]
	\label{Kirchhoff2.5D}
\end{multline} 

\autoref{Kirchhoff2.5D} is known as Kirchhoff 2.5D integral, because it aims at synthesizing a wave field on a two dimensional plane, but using 3D wave propagation equations.

\begin{shownto}{private}
\section*{Normalization of scenario}
Let's notice something that can actually be very useful when working with these expressions. Whatever the shape of the distribution of secondary sources is chosen to synthesize a field, if we express the distances not in meters but in wavelengths, we will see that the synthesized field can be expressed as an integral over that 'normalized' scenario when $\lambda = 1$, divided by the wavelength.

As $S_0$ is a line of length $L$, any point on that line can be identified by a single parameter $l$: $\PosTheo[section] = \PosTheo[section](l)$. Then, also the quantities that depend on that position can be expressed as a one-dimensional function: $\distLinePrimSource(l)$, $\distLinePoint(l)$, $\Diag(l)$, $\normPrimaryPropAngleSection(l)$. Now, define another scenario that is a spatially scaled version of the original, where distances are expressed in wavelengths and not in meters. That means the positions of sources and receiving points are transformed to normalized versions: $\PosTheo^{n} = \PosTheo/\lambda$, $\PosTheo[primarySource]^{n} = \PosTheo[primarySource]/\lambda$ and $\PosTheo[section](l)^{n} = \PosTheo[section](l\lambda)/\lambda$, $L^{n} = L/\lambda$. The derived parameters will also have a normalized equivalent: $\distLinePrimSource(l)^{n} = \distLinePrimSource(l\lambda)/\lambda$, $\distLinePoint(l)^{n} = \distLinePoint(l\lambda)/\lambda$, $\Diag^{n}(l) = \Diag(l\lambda)$, $\normPrimaryPropAngleSection^{n} = \normPrimaryPropAngleSection(l\lambda)$.

Let's see what happens when we use the monopole distribution:
\begin{multline}
	\widetilde{P_1}(\PosTheo) = \int_{S_0} \widetilde{I_1}(\PosTheo[section], \PosTheo) \dif[{\PosTheo[section]}] = \\ \int_{S_0} S(\omega) \Diag(\azimuth, 0, \omega) \cos\normPrimaryPropAngleSection \frac{e^{-jk\distLinePrimSource}}{\sqrt{\distLinePrimSource}} \frac{e^{-jk\distLinePoint}}{\distLinePoint} \sqrt{j2\pi k} \sqrt{\frac{\distLinePoint}{\distLinePrimSource + \distLinePoint}} \dif[{\PosTheo[section]}] = \\
	S(\omega) \int_{0}^{L} \Diag(l) \cos\normPrimaryPropAngleSection(l) \frac{e^{-jk\distLinePrimSource(l)}}{\sqrt{\distLinePrimSource(l)}} \frac{e^{-jk\distLinePoint(l)}}{\distLinePoint(l)} \sqrt{j2\pi k} \sqrt{\frac{\distLinePoint(l)}{\distLinePrimSource(l) + \distLinePoint(l)}} \dif[l] \\
    S(\omega) \lambda \int_{0}^{L/\lambda} \Diag(l\lambda) \cos\normPrimaryPropAngleSection(l\lambda) \frac{e^{-j2\pi\distLinePrimSource(l\lambda)/\lambda}}{\sqrt{\distLinePrimSource(l\lambda)/\lambda}\sqrt{\lambda}}\frac{e^{-j2\pi\distLinePoint(l\lambda)/\lambda}}{(\distLinePoint(l\lambda)/\lambda)\lambda} \sqrt{j/\lambda} \sqrt{\frac{\distLinePoint(l\lambda)/\lambda}{\distLinePrimSource(l\lambda)/\lambda + \distLinePoint(l\lambda)/\lambda}} \dif[l]
	\\
	= S(\omega) \frac{1}{\lambda} \int_{0}^{L^{n}} \Diag^{n}(l) \cos\normPrimaryPropAngleSection^{n}(l) \frac{e^{-j2\pi\distLinePrimSource^{n}(l)}}{\sqrt{\distLinePrimSource^{n}(l)}} \frac{e^{-j2\pi\distLinePoint^{n}(l)}}{\distLinePoint^{n}(l)} \sqrt{j} \sqrt{\frac{\distLinePoint^{n}(l)}{\distLinePrimSource^{n}(l) + \distLinePoint^{n}(l)}} \dif[l] 
\end{multline}

Moreover, this property not only works when scaling the scenario by $\lambda^{-1}$, but by any quantity. If the synthesized field in a given point of a scenario is $P(\PosTheo, \lambda)$ and we scale it by $C^{-1}$ (distances divided by $C$, included $\lambda^{n} = \lambda/C$), the resulting field $P_{s}(\PosTheo^{n}, \lambda^{n})$ is:
\begin{equation}
P_s(\PosTheo^{n}, \lambda^{n}) = C P(\PosTheo, \lambda)
\end{equation}
 
The same deductions can be applied with the secondary dipole distribution.
\end{shownto}