In this chapter, the mathematical basis of Wave Field Synthesis (WFS) is explained. First, we combine the monochromatic wave equation in three dimensions and Green's Theorem to prove that the wave field in a source-free space volume is totally determined by the wave field on its surface. Specifically, Kirchhoff integral and Rayleigh integral are the expressions that allow us to calculate the wave field at any point inside the volume using just its surface information, but Rayleigh integral is simpler and, hence, more convenient. We will base the rest of the WFS theory on it.

The main idea of WFS is to use loudspeakers to generate a surface acoustic wave field identical to the one that would be created by a virtual sound source. Since the wave field inside the volume depends only on the surface one, it will be the same as the acoustic field that would be generated by the virtual source. The accuracy of WFS depends on how well we replicate the surface wave field of the virtual source. We will see that we would need an infinite amount of monopole and dipole infinitesimal sources to generate an exact acoustic field, but it is obvious that in any real situation we can only use a finite number of loudspeakers that are not infinitesimal, nor they present ideal monopole or dipole radiation patterns. 

So, we will model a WFS system with discrete punctual sources and take a look at the problems that derive from it.



\section{Kirchhoff and Rayleigh integrals}

The Kirchhoff integral finds the solution to the homogeneous wave equation at an arbitrary point $\mathbf{p}$ in terms of the values of the solution of the wave equation and its first-order derivative at all points on an arbitrary surface that encloses $\mathbf{p}$.

In the acoustic field, this means that the sound pressure in any point of a source-free volume can be calculated in both the sound pressure and its gradient are known on the surface enclosing the volume. 

Rayleigh integral is a special case of Kirchhoff integral. Let's see how these equations are calculated.

Acoustic pressure propagates through the air as a wave. That means it satisfies the wave equation:

\begin{equation}
\Delta P - \frac{1}{c^2}\frac{\partial^2 P}{\partial t^2} = 0
\label{eqWave}
\end{equation}
where
\begin{description}
	\item[$P$] Acoustic pressure
	\item[$c$] Propagation velocity
\end{description}

If we only consider one frequency, the field can be expressed as $P = u(\mathbf{x})e^{j 2\pi f t}$, where $u$ is called wave field. Then, Eq.\ref{eqWave} transforms to Helmholtz equation:

\begin{equation}
\Delta u - k^2 u = 0
\label{eqHelmholtz}
\end{equation}
where
\begin{description}
	\item[$k$] Wavenumber. $k = 2\pi f/c$
\end{description}

On the other hand, Green’s theorem, which is a generalization of integration by parts, states that:

\begin{theorem}
	\label{theoGreen}
	For a domain $V \in \mathbb{R}^3$, and $u$ and $v$ sufficiently regulars, we have
	\begin{equation}
	\int_{V} \left( u \Delta v - v \Delta u \right) dV = \int_{S} \left(u\frac{\partial v}{\partial \mathbf{n}} - v \frac{\partial u}{\partial \mathbf{n}}\right) dS
	\label{eqGreen}
	\end{equation}
	where $\mathbf{n}$ is the outward-pointing unit normal to the boundary $S$ of $V$.
\end{theorem}

If functions $u$ and $v$ satisfy Helmholtz equation (Eq.\ref{eqHelmholtz}), the integral described in theorem.\ref{theoGreen} becomes 0:
\begin{equation}
	\int_{V} \left( u \Delta v - v \Delta u \right) dV = \int_{V} \left(u k^2 v - v k^2 u \right) dV = 0 = \int_{S} \left(u\frac{\partial v}{\partial \mathbf{n}} - v \frac{\partial u}{\partial \mathbf{n}}\right) dS
	\label{eqWaveAndGreen}
\end{equation}

Let $v(\vec{x}) = G(\vec{x} - \vec{x_0})$ where:
\begin{equation}
G(\vec{x}) = \frac{e^{-j k \norm{\vec{x}} }}{\norm{\vec{x}}}
\label{eqGreensFunction}
\end{equation}
and $\vec{x_0} \in V$.
We have a problem: $v$ satisfies Helmholtz equation, but has a singularity at $\vec{x_0}$, so it is not smooth enough for Green's Theorem to be applied. The solution is to remove from the volume $V$ a small sphere around the singularity point. The centre is precisely $\vec{x_0}$ and the radius can be as small as possible as long as it is bigger than 0: $\delta > 0$. We will denote the sphere as $B(\vec{x_0}, \delta)$. The new volume is defined as $V_{\delta} = V \\ B(\vec{x_0}, \delta)$ and then, now there are two surfaces of integration. One of them is the original one, and the other is the surface of the sphere $\partial B(\vec{x_0}, \delta)$. Green's Theorem transforms to:

\begin{equation}
 \int_{V} \left( u \Delta v - v \Delta u \right) dV = \int_{S} \left(u\frac{\partial v}{\partial \mathbf{n}} - v \frac{\partial u}{\partial \mathbf{n}}\right) dS + \int_{\partial B(\vec{x_0}, \delta)} \left(u\frac{\partial v}{\partial \mathbf{n}} - v \frac{\partial u}{\partial \mathbf{n}}\right) dS
\end{equation}

In any point of the surface $\partial B(\vec{x_0}, \delta)$, $\vec{n} = \frac{\vec{x_0} - \vec{x}}{\norm{\vec{x_0} - \vec{x}}} = \frac{\vec{x_0} - \vec{x}}{\delta}$. That implies that the normal derivative of $v(\vec{x})$ is the same on the whole surface:

\begin{multline}
\frac{\partial v(\vec{x})}{\partial \vec{n}} \rvert_{\vec{x} \in \partial B(\vec{x_0}, \delta)} = \langle \vec{n} , \nabla v(\vec{x}) \rangle =
\Big\{ \nabla v(\vec{x}) = G(\vec{x} - \vec{x_0})(-jk - \frac{1}{\norm{\vec{x} - \vec{x_0}}})\frac{\vec{x} - \vec{x_0}}{\norm{\vec{x} - \vec{x_0}}}|_{\vec{x} \in \partial B(\vec{x_0}, \delta)}= \\
G(\vec{x} - \vec{x_0})(-jk - \frac{1}{\delta})\frac{\vec{x} - \vec{x_0}}{\delta} \Big\} = \\ G(\vec{x} - \vec{x_0})(-jk - \frac{1}{\delta})\langle \frac{\vec{x} - \vec{x_0}}{\delta} , \frac{\vec{x_0} - \vec{x}}{\delta} \rangle = G(\vec{x} - \vec{x_0})\left( jk + \frac{1}{\delta} \right)
\end{multline}

The surface integral of the sphere when $\delta \rightarrow 0$ becomes:
\begin{multline}
\int_{\partial B(\vec{x_0}, \delta)} \left(u\frac{\partial v}{\partial \mathbf{n}} - v \frac{\partial u}{\partial \mathbf{n}}\right) dS = \int_{\partial B(\vec{x_0}, \delta)} G(\vec{x} - \vec{x_0}) \left( u(\vec{x})(jk - 1/\delta) - \frac{\partial u}{\partial \vec{n}} \right) dS = \\
\frac{e^{-jk\delta}}{\delta}\delta^2 \int_{\partial B(\vec{0}, 1)} \left(u(\vec{x_0} + \delta\vec{x})(jk - 1/\delta) - \frac{\partial u}{\partial \vec{n}} \right) dS \approx 4\pi u(\vec{x_0})
\label{eqBallIntegral}
\end{multline}

Hence, from Eq.\ref{eqWaveAndGreen} and Eq.\ref{eqBallIntegral} we conclude that:
\begin{equation}
u(\vec{x_0}) = \frac{1}{4\pi} \int_{S} \left(G(\vec{x} - \vec{x_0}) \frac{\partial u(\vec{x})}{\partial \mathbf{n}} - u(\vec{x})\frac{\partial G(\vec{x} - \vec{x_0})}{\partial \mathbf{n}} \right) dS
\label{eqKirchhoffIntegral}
\end{equation}
Previous equation is what is called Kirchhoff integral.

We can interpret it as if each surface element was emitting two different radiation patterns. The first one would be the corresponding to the first term of Eq.\ref{eqKirchhoffIntegral}:

\begin{equation}
G(\vec{x} - \vec{x_0}) \frac{\partial u(\vec{x})}{\partial \mathbf{n}}
\end{equation}

Each surface element radiates like a monopole with an amplitude given by $\frac{\partial u(\vec{x})}{\partial \mathbf{n}} dS$.

The second contribution is more complex:
\begin{equation}
\begin{aligned}
&u(\vec{x})\frac{\partial G(\vec{x} - \vec{x_0})}{\partial \mathbf{n}} = u(\vec{x})\langle \vec{n} , \nabla G(\vec{x} - \vec{x_0}) \rangle\\
&\text{where:}\\
&\nabla G(\vec{x} - \vec{x_0}) = G(\vec{x} - \vec{x_0})(-jk - \frac{1}{\norm{\vec{x} - \vec{x_0}}})\frac{\vec{x} - \vec{x_0}}{\norm{\vec{x} - \vec{x_0}}} = \\
&\downarrow \\
&\frac{\partial G(\vec{x} - \vec{x_0})}{\partial \mathbf{n}} = G(\vec{x_0} - \vec{x})\left( jk + \frac{1}{\norm{\vec{x_0} - \vec{x}}} \right) \langle \frac{\vec{x_0} - \vec{x}}{\norm{\vec{x_0} - \vec{x}}} , \vec{n} \rangle
\end{aligned}
\end{equation}

The amplitude is simpler because it is actually the value of $u(\vec{x})$ at each point of the surface. The radiation pattern is much more complex though. We can see it behaves like a dipole, with the difference that it is multiplied by a scalar that depends on the distance: $jk + \frac{1}{\norm{\vec{x_0} - \vec{x}}}$.

Kirchhoff integral can be simplified by making some assumptions. We need to assume:

- The volume $V$ is a semisphere of infinite radius.

- Sommerfeld’s radiation condition, which means that we are only dealing with outgoing waves, i.e., there are no sources at infinity. Mathematically, this can be expressed by:
\begin{equation}
\lim_{\norm{\vec{x}}\to\infty} \norm{\vec{x}} \left( -jku(\vec{x}) - \frac{\partial u}{\partial \vec{n}}\rvert_{\vec{x}} \right) = 0
\end{equation}

Then, the contribution of the surface of the sphere becomes null and we only have to take in account the field in the plane.

\begin{equation}
u(\vec{x_0}) = \frac{1}{4\pi} \int_{A} \left(G(\vec{x} - \vec{x_0}) \frac{\partial u(\vec{x})}{\partial \mathbf{n}} - u(\vec{x})\frac{\partial G(\vec{x} - \vec{x_0})}{\partial \mathbf{n}} \right) dS
\end{equation}

We can go farther and use another Green's function to eliminate one of the terms of the Kirchhoff integral. This new Green function is:

\begin{equation}
g_{\vec{x}_0} = G(\vec{x} - \vec{x_0}) - G(\vec{x} - \overline{\vec{x}}_0)
\end{equation}

where $\overline{\vec{x}}_0$ is the reflection of $\vec{x}_0$ respect to the plane of the semisphere. The value of $g_{\vec{x}_0}$ becomes $0$ in the aperture plane. So, we can apply the previous calculations to obtain:
\begin{equation}
u(\vec{x_0}) = \frac{-1}{4\pi} \int_{A} u(\vec{x})\frac{\partial g_{\vec{x}_0}(\vec{x} - \vec{x_0})}{\partial \mathbf{n}} dS = \frac{-2}{4\pi} \int_{A} u(\vec{x})G(\vec{x} - \vec{x_0})\left( -jk - \frac{1}{\norm{\vec{x} - \vec{x_0}}}\right) \langle \frac{\vec{x} - \vec{x_0}}{\norm{\vec{x} - \vec{x_0}}} , \vec{n} \rangle dS
\label{eqRayleigh}
\end{equation}
Eq.\ref{eqRayleigh} is called Rayleigh integral, where only dipole sources are considered.

% According to \cite{Berkhout1993}:

%\begin{multline}
%	P(\mathbf{r}) = \frac{1}{4\pi}\int_{S} \left[P(\mathbf{r_s})\frac{\partial G}{\partial n} - G\frac{\partial P(\mathbf{r_s})}{\partial n}\right] dS = 
%	\frac{1}{4\pi}\int_{S} [P(\mathbf{r_s})(\hat{n}\cdot\nabla G) - G(\hat{n}\cdot\nabla P(\mathbf{r_s}))] dS \\ \Bigg\{ 
%	G = \frac{e^{-jk|\mathbf{r}-\mathbf{r_s}|}}{|\mathbf{r}-\mathbf{r_s}|}, \qquad
%	\nabla G = \nabla \frac{e^{-jk|\mathbf{r}-\mathbf{r_s}|}}{|\mathbf{r}-\mathbf{r_s}|} = \frac{e^{-jk|\mathbf{r}-\mathbf{r_s}|}}{|\mathbf{r}-\mathbf{r_s}|^2}\left(-jk-\frac{1}{|\mathbf{r}-\mathbf{r_s}|}\right)(\mathbf{r_s}-\mathbf{r})
%	\Bigg\}
%	\label{eqKirchhoffIntegral2}
%\end{multline}
%
%
%For a punctual source located at $\mathbf{r_0}$:
%\begin{multline}
%P(\mathbf{r}) = \frac{1}{4\pi}\int_{S} [P(\mathbf{r_s})(\hat{n}\cdot\nabla G) - G(\hat{n}\cdot\nabla P(\mathbf{r_s}))] dS \\= \Bigg\{ \nabla P(\mathbf{r_s}) = \nabla \frac{A e^{-jk|\mathbf{r}-\mathbf{r_0}|}}{|\mathbf{r_s}-\mathbf{r_0}|} = A\frac{e^{-jk|\mathbf{r_s}-\mathbf{r_0}|}}{|\mathbf{r_s}-\mathbf{r_0}|^2}\left(-jk-\frac{1}{|\mathbf{r_s}-\mathbf{r_0}|}\right)(\mathbf{r_s}-\mathbf{r_0})
%\Bigg\} \\
%= \frac{1}{4\pi}\int_{S}\left[\frac{Ae^{-jk|\mathbf{r}-\mathbf{r_0}|}}{|\mathbf{r_s}-\mathbf{r_0}|}\frac{e^{-jk|\mathbf{r_s}-\mathbf{r}|}}{|\mathbf{r_s}-\mathbf{r}|}\left(jk\left(\frac{\mathbf{r} - \mathbf{r_s}}{|\mathbf{r} - \mathbf{r_s}|} + \frac{\mathbf{r_s} - \mathbf{r_0}}{|\mathbf{r_s} - \mathbf{r_0}|}\right)
%+ \frac{\mathbf{r_s} - \mathbf{r_0}}{|\mathbf{r_s} - \mathbf{r_0}|^2}
%+ \frac{\mathbf{r} - \mathbf{r_s}}{|\mathbf{r} - \mathbf{r_s}|^2}\right)\cdot\hat{n} \right] dS
%\end{multline}
%
%If the surface S degenerates es to a semisphere of infinite size, we can consider that there's a plane that separates the listening area from the primary source area (Fig.\ref{semisphereRayleigh}). In that case, te integral of the first term of the Kirchhoff's integral (Eq.\ref{eqKirchhoffIntegral2}) becomes equal to the integral of the second term. Hence, the pressure can be calculated with Rayleigh I or II integral. In \cite{Berkhout1993} the Rayleigh II integral is used:
%
%\begin{equation}
%P(\mathbf{r}) = \frac{1}{2\pi}\int_{S} P(\mathbf{r_s})\frac{\partial G}{\partial n} dS = \frac{1}{2\pi}\int_{S} \Big[P(\mathbf{r_s})\frac{e^{-jk|\mathbf{r}-\mathbf{r_s}|}}{|\mathbf{r}-\mathbf{r_s}|^3}\left(1+jk|\mathbf{r}-\mathbf{r_s}|\right)(\mathbf{r}-\mathbf{r_s})\cdot \hat{n}\Big] dS = 
%\label{eqRayleighII}
%\end{equation}
