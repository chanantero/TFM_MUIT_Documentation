\section{Introduction}
In the GTAC anechoic chamber there is a 96-loudspeaker array distributed as an irregular octagon in the horizontal plane (\autoref{WFSdistribution}), $1.65 \si{m}$ above the floor, with a separation between loudspeakers of $\secondarySourceSeparation = 0.18 \si{m}$. 

Many variables have an impact on the acoustic field generated by the loudspeakers: frequency dependent directivity of each individual loudspeaker, non-linearities, reverberation of the chamber, diffraction, reflections on the floor (which is not recovered with absorbing material as the walls), etc. However, if we assume a simple model where the anechoic chamber is perfectly configured to emulate free-space conditions, and every loudspeaker is identical to the rest and behaves as an ideal monopole, then the similarities with the scenario presented by WFS theory become clear.

Under ideal conditions, the octagon can be interpreted as the closed curve of secondary sources $\sectionTheo$ discretized with a step of $\secondarySourceSeparation = 0.18 \si{m}$. As we only count with monopole sources (loudspeakers with dipole characteristics are more difficult and expensive to manufacture), we should use the Rayleigh 2.5D I integral (\autoref{RayleighI2.5}), but $\sectionTheo$ should be an infinite line and not an octagon. Of course, in practice an infinite array is not realizable, so at some point we must truncate the array anyway. On the other hand, when dealing with a bent array, the amplitude factor $g = \sqrt{\frac{d}{d + \abs{\PosTheo[primarySource][z]}}}$ that was calculated when $\sectionTheo$ was a straight line might not be the best option any more.

The actual loudspeaker feeding signals that were used, were calculated applying formulas that were provided by the professors and are particularized for the specific geometry of the GTAC array:
\begin{equation}
\begin{aligned}
g &= 
\begin{cases}
\sqrt{\frac{\distLinePoint}{\distLinePrimSource + \distLinePoint}} & \normPrimaryPropAngle \leq 90^\circ \\
0 & \normPrimaryPropAngle > 90^\circ
\end{cases}
\\
\distLinePoint &= \frac{1.44}{2} + 1.44 \cos\left( \frac{\pi}{4} \right)
\end{aligned}
\label{amplitFactorGTAC}
\end{equation}

\section{Simulation 1}
Traditionally, WFS has been used, not to cancel noise, but as a spatial audio reproduction system that competes with existing stereophonic systems as Dolby Surround. The main focus has been, then, not in replicating accurately a field, but in generating the subjective impression of natural hearing, this is, of sound heard from various directions. So, the evaluation of performance has been usually guided by the ability of subjects to localize virtual sound sources and other subjective measures.

Since human hearing has limitations, there are objective sound characteristics that it cannot perceive. We can take advantage of this, and use compression, downsampling and other techniques (common in mp3 and other compression formats) that lower the requirements of the system without worsening the subjective perception. One thing that humans cannot distinguish is
