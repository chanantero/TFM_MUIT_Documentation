%\section{Rayleigh 2.5D I and II integrals}
Given that planar loudspeaker arrays are less convenient than linear arrays in practical situations, a dimensionality reduction from 3D to 2D is desirable, as it is proposed in \cite{Vogel} and developed in \cite{Start1997} and  \cite{Verheijen}.
% and \cite{stuart1996application}.
Raileigh I integral (\autoref{RaileighI}) can be transformed from a surface integral of the plane $S_1$ ($z=0$), to a line integral of the $x$ axis. In order to do that, we divide surface $S_1$ in the 

we cut $S_1$ in the $\hat{x}$ direction

as a set of lines that extend in the $\hat{y}$ direction, put one next to the other We perform the dimensionality reduction by calculating the contribution of each of those lines.

Then, the field is expressed not as a surface integral, but as a line integral over

\begin{equation}
\begin{aligned}
P(\PosTheo) &= \frac{-1}{2\pi} \int_{S_1} \GreenFunc[{\PosTheo - \PosTheo[surface]}] \frac{\partial P(\PosTheo[surface])}{\partial \mathbf{n}} dS = \int_{-\infty}^{\infty} I(x_s, \PosTheo) dx_s \\
I(x_s, \PosTheo) &= \frac{-1}{2\pi} \int_{-\infty}^{\infty} \GreenFunc[{\PosTheo - \PosTheo[surface]}] \frac{\partial P(\PosTheo[surface])}{\partial \mathbf{n}} dy_s
%I(x) = \int_{-\infty}^{\infty} \frac{e^{-jk\sqrt{(x - x_0)^2 + y^2 + (z-z_0)^2}}}{\sqrt{(x - x_0)^2 + y^2 + (z-z_0)^2}} \frac{\partial P(\vec{x})}{\partial \mathbf{n}} dy
\end{aligned}
\label{LineIntegral}
\end{equation}

\begin{shownto}{private}
So, the question is how to calculate $I(x_s, \PosTheo)$. We start by calculating the directional derivative of the pressure field. We assume it is generated by a primary source located at $\PosTheo[primarySource]$, and that the produced field can be modelled as $P(\PosTheo) = \CoefTheo \Diag(\phi, \theta) \GreenFunc[{\PosTheo - \PosTheo[primarySource]}]$, where $\CoefTheo$ is the source coefficient and $\Diag(\phi, \theta)$ is the directivity pattern ($\phi = \arctan(x/z)$, $\theta = \arctan(y/\sqrt{x^2 + z^2})$). The directional derivative is then \cite{Verheijen}:
\begin{multline}
\frac{\partial P(\PosTheo[surface])}{\partial \mathbf{n}} = \frac{\partial \coef \Diag(\phi, \theta) \GreenFunc[{\PosTheo[surface] - \PosTheo[primarySource]}]}{\partial \mathbf{z}} = \\ -A \GreenFunc[{\PosTheo[surface] - \PosTheo[primarySource]}] \left[ \frac{\sin\phi}{\norm{\PosTheo[surface] - \PosTheo[primarySource]}\cos\theta}\frac{\partial\Diag}{\partial\phi} + \frac{\cos\phi \sin\theta}{\norm{\PosTheo[surface] - \PosTheo[primarySource]}} \frac{\partial\Diag}{\partial\theta} + \left(jk + \frac{1}{\norm{\PosTheo[surface] - \PosTheo[primarySource]}}\right) \Diag \cos\phi \cos\theta \right]
\label{derivPressure}
\end{multline}

%Inserting \autoref{derivPressure} in \autoref{LineIntegral} we get:
%\begin{multline}
%I(x_s, \PosTheo) = -A \int_{-\infty}^{\infty} \GreenFunc[{\PosTheo - \PosTheo[surface]}] \GreenFunc[{\PosTheo[surface] - \PosTheo[primarySource]}] \cross
%\\
%\cross \left[ \frac{\sin\phi}{\norm{\PosTheo[surface] - \PosTheo[primarySource]}\cos\theta}\frac{\partial\Diag}{\partial\phi} + \frac{\cos\phi \sin\theta}{\norm{\PosTheo[surface] - \PosTheo[primarySource]}} \frac{\partial\Diag}{\partial\theta} + \left(jk + \frac{1}{\norm{\PosTheo[surface] - \PosTheo[primarySource]}}\right) \Diag \cos\phi \cos\theta \right] dy_s
%\label{integrateY}
%\end{multline}

Now, since we intend a dimensionality reduction for practical purposes, it's reasonable to set the constraint that all primary sources and the positions where we want to synthesize their field are all located in the same plane, for example, $y = 0$. This means that $\PosTheo = [x, 0, z],\quad z > 0$ and $\PosTheo[primarySource] = [x_\mathit{ps}, 0, z_\mathit{ps}], \quad z_\mathit{ps} < 0$.

According to the stationary phase method \cite{Verheijen},
\begin{equation}
I = \int_{-\infty}^{+\infty} f(y) e^{j\phi(y)} dy \approx f(y_0) e^{j\phi(y_0)} \sqrt{\frac{j 2\pi}{\phi''(y_0)}}
\end{equation}
where $y_0$ is the value of $y$ where the phase gets stationary $\frac{d\phi(y)}{dy} = 0$, and $\phi''(y_0)$ is the second derivative of $\phi$ evaluated at $y_0$. Intuitively, it means that the integral of a phase changing function only has a significant contribution in the region where the phase change slows down.

Applying this method to the combination of \autoref{derivPressure} and \autoref{LineIntegral}, we get that
\begin{gather}
	y_0 = 0 \\
	\phi(y_0) = -k(\distLinePrimSource + \distLinePoint) \\
	\phi''(y_0) = -k\frac{\distLinePrimSource + \distLinePoint}{\distLinePrimSource \distLinePoint} \\
	f(y_0) = -\frac{A}{\distLinePrimSource \distLinePoint}\left[ \frac{\sin\phi}{\distLinePrimSource} \frac{\partial\Diag}{\partial\phi} + \left(jk + \frac{1}{\distLinePrimSource}\right) \Diag(\phi, 0) \cos\phi \right]
\end{gather}
where $\distLinePrimSource = \norm{\PosTheo[surface] - \PosTheo[primarySource]}\Big\vert_{y_{\PosTheoSubInd[surface]} = 0} = \sqrt{z_{\PosTheoSubInd[primarySource]}^2 + (x_{\PosTheoSubInd[surface]} - x_{\PosTheoSubInd[primarySource]})^2}$ and $\distLinePoint = \norm{\PosTheo[surface] - \PosTheo}\Big\vert_{y_{\PosTheoSubInd[surface]} = 0} = \sqrt{z^2 + (x_{\PosTheoSubInd[surface]} - x)^2}$.

Assuming $k\distLinePrimSource >> 1$
\end{shownto}
\begin{shownto}{public}
Assuming that the field generated at location $\PosTheo$ by a primary monopole source located at $\PosTheo[primarySource]$ is 
\begin{equation}
P(\PosTheo) = S \GreenFunc[{\PosTheo - \PosTheo[primarySource]}]
\end{equation}
and employing the stationary phase method and other approximations ($k\distLinePrimSource >> 1$), the result is \cite{Start1997}:
\end{shownto}
\begin{equation}
I(x_{\PosTheoSubInd[surface]}, \PosTheo) \approx \widetilde{I}(x_{\PosTheoSubInd[surface]}, \PosTheo) = S \frac{e^{-jk\distLinePoint}}{\distLinePoint} \cos\normPrimaryPropAngleSection \frac{e^{-jk \distLinePrimSource}}{\sqrt{\distLinePrimSource}} \sqrt{\frac{jk}{2\pi}} \sqrt{\frac{\distLinePoint}{\distLinePrimSource + \distLinePoint}}
\end{equation}
\begin{shownto}{public}
where $\distLinePoint = \norm{\PosTheo[surface] - \PosTheo[primarySource]} = \sqrt{}$ is the distance between the primary source and the point at the line
\end{shownto}

Let's see that in previous equation there is the expression for the propagation of a monopole. So, we can express it as the strength of a secondary monopole source $Q_m(x_s, \PosTheo)$ propagated to the receiving point $\PosTheo$.
\begin{equation}
\begin{aligned}
\widetilde{I}(x_{\PosTheoSubInd[surface]}, \PosTheo) &= Q_m(x_{\PosTheoSubInd[surface]}, \PosTheo) \frac{e^{-jk\distLinePoint}}{\distLinePoint} \\
Q_m(x_{\PosTheoSubInd[surface]}, \PosTheo) &= S \cos\normPrimaryPropAngleSection \frac{e^{-jk\distLinePrimSource}}{\sqrt{\distLinePrimSource}}\sqrt{\frac{jk}{2\pi}} \sqrt{\frac{\distLinePoint}{\distLinePrimSource + \distLinePoint}}
\end{aligned}
\end{equation}
Finally, the field is:
\begin{equation}
P(\PosTheo) = \int_{-\infty}^{\infty} Q_m(x_{\PosTheoSubInd[surface]})
\frac{e^{-jk\distLinePoint}}{\distLinePoint} dx_{\PosTheoSubInd[surface]}
\label{RayleighI2.5}
\end{equation}

\autoref{RayleighI2.5} is known as Rayleigh I 2.5D integral, because it aims at synthesizing a wave field on a plane, but using 3D wave propagation equations. However, it has the problem that the amplitude of the secondary monopole sources depends on the receiver position, which is impossible to realize in practice.
This can be solved by modifying the amplitude factor $g = \sqrt{\frac{\distLinePoint}{\distLinePrimSource + \distLinePoint}}$.
Applying the stationary phase method, but now in the x direction, we would find that the main contribution to the pressure at a given point comes from the intersection of the secondary source line and the line that goes from the primary source to the receiver point. Taking advantage of this, we can change $\distLinePoint$ by $d$ and $\distLinePrimSource$ by $\vert z_\mathit{ps} \vert$, so $g = \sqrt{\frac{d}{d+\vert z_{\PosTheoSubInd[primarySource]} \vert}}$. Therefore, $Q(x)$ becomes independent from the receiver point (implementable in real WFS systems) and the pressure will be rightly scaled at a line of receivers parallel to the secondary source line, and separated a distance $d$. For other values of $d$, there will be amplitude errors \cite{Verheijen}. 
