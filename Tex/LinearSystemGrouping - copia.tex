%\chapter{Linear System Grouping}

%Grouping a linear system is an operation that consists of solving it with a restriction. It consists in that the unknowns aren't the coefficients anymore, but a series of parameters that multiply groups of the default coefficients.
%
%Let's say we have an linear system \aurtoref{normalLinearSystem}.

%\begin{equation}
%\myMatrix{A} \vec{x} = \vec{y}
%\end{equation}

%Now we modify it by defining an initial solution $\vec{x}_0$ and generating the next equation:
%
%\begin{equation}
%\myMatrix{A'} \vec{x'} = \vec{y}
%\end{equation}
%
%\begin{equation}
%\myMatrix{A'} = [\vec{a_1^'}, \vec{a_n^'}, \vec{a_N^'}]
%\end{equation}
%
%where
%
%\begin{equation}
%\vec{a_n^'} = \sum_{i \in I_n} x_i \vec{a}_i
%\end{equation}
%
%and $I_n$ is a set of indices.
%Each index $i$ maps onto the columns of $A$, hence $i \in \{1, ..., L\}$.
%
%Every element of $I_n$ is a partition of the set of the indices of the columns of $\myMatrix{A}$ ($S = \{1, ..., L\}$).
%In other words, each element of $S$/column index belongs to, at most, only one set $I_n$ / one group.
%
%$P = {I_1, I_2, ..., I_n,... I_N}$ is an indexed family of sets. % https://www.whitman.edu/mathematics/higher_math_online/section01.06.html
%
%All sets $I_n$ are disjoint, 
%
%the intersection of any two different sets is the empty space
%
%Two sets $I_{n_1}$ and $I_{n_2}$ don't have any element in common.
%
%\begin{equation}
%	I_{i} \cup I_{j}
%\end{equation}
%
%\begin{equation}
%\left( \bigcup_{n \in } I_{n} \right) \in S
%\end{equation}
%
%
%
