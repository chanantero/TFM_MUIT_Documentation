\chapter{Linear System Grouping}

Let's say we have an linear system (\autoref{normalLinearSystem}) and we are trying to find the solution $\vec{x}$:

\begin{equation}
\myMatrix{A}_{MxL} \vec{x}_{Lx1} = \vec{y}_{Mx1}
\label{normalLinearSystem}
\end{equation}

We divide the columns into $N$ different groups. Each one of them is defined by a set of indices $I$ that map onto the indices of the columns of $\myMatrix{A}$ ($I \subseteq \{1, \ldots, L\}$). In other words, the n-th group contains the columns with the indices contained in $I_n$. For example, if we define two groups of columns, $I_1 = \{1, 3, 4\}$ and $I_2 = \{2, 5\}$, the first group will contain the \nth{1}, \nth{3} and \nth{4} columns, and the second group the \nth{2} and \nth{5} columns. Each column can only be included in one group, this is, there isn't a column index that is contained in more than one set $I_n$ ($I_p \cap I_q = \emptyset, \forall p, q \text{ such that } p \neq q \text{ and } \{p, q\} \subseteq \{1, \ldots, N\}$). We also define a vector of default values $\vec{x}_0$.

A new matrix $\myMatrix{A}'_{MxN}$ is created. The n-th column is the matrix multiplication of the columns of $\myMatrix{A}$ that belong to the n-th group by the correspondent elements of $\vec{x}_0$.

\begin{equation}
\begin{aligned}
\myMatrix{A}' &= [\vec{a}'_1,\ldots, \vec{a}'_n, \ldots, \vec{a}'_N] \\
\vec{a}'_n &=
%\left. \begin{cases}
%\myMatrix{A}_n = [\vec{a}_{I_{n(1)}}, \ldots, \vec{a}_{I_{n(|I_n|)}}] \\
%\vec{x}_{0n} = [\vec{x}_{0(I_{n(1)})}, \ldots, \vec{x}_{0(I_{n(|I_n|)})}]
%\end{cases} \right\} = \myMatrix{A}_n \vec{x}_{0n} =
\sum_{i \in I_n} x_{0(i)} \vec{a}_i
\end{aligned}
\end{equation}

Now, we transform the left side of previous linear system (\autoref{normalLinearSystem}) to the next one with a new unknown $\vec{x}'$:

\begin{equation}
\myMatrix{A}'_{MxN} \vec{x}'_{Nx1} = \vec{y}_{Mx1} - \vec{y}_c
\label{modifiedLinearSystem}
\end{equation}

We've introduced also the vector $\vec{y}_c$, which is the result of the matrix multiplication of the columns of $\myMatrix{A}$ that do not belong to any group by the correspondent elements of $\vec{x}_0$.

\begin{equation}
\vec{y}_c = \sum_{i \in C} x_{0(i)} \vec{a}_i , \qquad
C = \{1, \ldots, L\} \setminus \bigcup_{n=1}^N I_n
\end{equation}

When the system (\autoref{modifiedLinearSystem}) is solved, whether it is by finding the exact solution or by finding the least squares solution, it can be interpreted as

we found the coefficients that multiply the default vector 

we set the constraint that coefficients inside groups should maintain a relative relation between them, a relation described by $x_0$.

So, we see that $x_0$ is useful when you want to establish a relative scale restriction between values of x


When the system (\autoref{modifiedLinearSystem}) is solved, whether it is by getting the exact solution or by finding the least squares solution in case the system is inconsistent, it can be interpreted as

we found the coefficients that multiply the default vector 

It is as we set the constraint that the values of $\vec{x}_0$ inside a group should maintain a relative relation between them, a relation given by $x_0$.

To map the solution to the solution of the original equation...

So we found the solution given the constraint described above.

A nomenclature one can use is...

%% Complete version, only for the author
%\begin{itemize}
%	\item $I_n$ is a set of indices that map onto the columns of $\myMatrix{A}$ ($i \in \{1, ..., L\}, \forall i \in I_n$).
%\end{itemize}
%
%\begin{equation}
%\bigcup_{n = 1}^{N} I_{n} \subseteq S
%\end{equation}
%
%\begin{itemize}
%	\item A given column is only contained in one group.
%	\item Each element of $S = \{1,\ldots,L\}$ belongs to, at most, only one set $I_n$.
%	\item All index groups $I_n$ are disjoint (their intersection is the empty space)
%	\item There isn't a column index that is present in more than one set $I_n$.
%	\item There aren't two sets $I_n$ that have elements in common.
%\end{itemize}
%
%\begin{equation}
%I_p \cap I_q = \emptyset, p \neq q, \{p, q\} \subseteq \{1, \ldots, N\}
%\end{equation}
%
% $P = \{I_1, I_2, ..., I_n,... I_N\}$ is an indexed family of sets. % https://www.whitman.edu/mathematics/higher_math_online/section01.06.html
% In case all columns are included in the union of the sets of $P$ ($\bigcup_{n = 1}^{N} I_{n} = S$), we can say that $P$ is a partition of $S$.
%
%\begin{equation}
%\begin{aligned}
%\myMatrix{A}' &= [\vec{a}'_1,\ldots, \vec{a}'_n, \ldots, \vec{a}'_N] \\
%\vec{a}'_n &= \left. \begin{cases}
%\myMatrix{A}_n = [\vec{a}_{I_{n(1)}}, \ldots, \vec{a}_{I_{n(|I_n|)}}] \\
%\vec{x}_{0n} = [\vec{x}_{0(I_{n(1)})}, \ldots, \vec{x}_{0(I_{n(|I_n|)})}]
%\end{cases} \right\} = \myMatrix{A}_n \vec{x}_{0n} = \sum_{i \in I_n} x_{0(i)} \vec{a}_i
%\end{aligned}
%\end{equation}
%
%\begin{gather}
%	\myMatrix{A}'_{MxN} \vec{x}'_{Nx1} = \vec{y}_{Mx1} - \vec{y}_c\\
%	\vec{y}_c = \left. \begin{cases}
%		\myMatrix{A}_c = [\vec{a}_{C_1}, \ldots, \vec{a}_{C_{|C|}}] \\
%		\vec{x}_c = [\vec{x}_{C_1}, \ldots, \vec{x}_{C_{|C|}}]
%	\end{cases} \right\}
%= \myMatrix{A}_c \vec{x}_c = 
%\sum_{i \in C} x_{0(i)} \vec{a}_i , \qquad
%	C = \{1, \ldots, L\} \setminus \bigcup_{n=1}^N I_n
%\end{gather}