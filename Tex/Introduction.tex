\section{Wave Field Synthesis}
Wave Field Synthesis (WFS) is a method that, by means of an array of loudspeakers (large number of small and closely spaced loudspeakers) reproducing the proper audio signals, generates the acoustic wave field that a hypothetical source of sound would produce (virtual primary source). In other words, it is a way of accurately replicating temporal, spectral and spatial properties of a sound field. For example, in a room where one of this arrays is set up, a person situated in any point of the room could hear the voice of a person moving through the room, as if someone that is not there was actually talking and walking.

\begin{figure}[h]
	\centering
	\begin{subfigure}[c]{0.45\textwidth}
		\centering
		\includegraphics[width=0.9\columnwidth]{Img/WFSconceptReal.pdf}
	\end{subfigure}
	\begin{subfigure}[c]{0.45\textwidth}
		\centering
		\includegraphics[width=0.9\columnwidth]{Img/WFSconceptVirtual.pdf}
	\end{subfigure}
\caption[WFS explanation]{WFS explanation \cite{icons1}}
\label{WFSimageExplanation}
\end{figure}

WFS takes advantage of a physical principle applied to wave fields (such as acoustic waves) expressed in Kirchhoff's integral. Before getting into the details of mathematical expressions, let's just say that Kirchhoff's integral states that in a homogeneous wave propagation media, in any source-free volume $\volumeTheo$ (fictive) delimited by a surface $\surfaceTheo$, the wave field at any point in that space can be calculated if the wave field and its gradient on the surface are known. In other words, if we want to know the sound that one can hear at any point inside $\volumeTheo$, we just have to measure the acoustic pressure and its gradient on $\surfaceTheo$.

\begin{figure}[h]
		\centering
		\def\svgwidth{0.5\columnwidth}
		\graphicspath{{../TFM/Img/}}
		\input{../TFM/Img/KirchhoffTheoSchemeIncomplete.pdf_tex}
	\caption[Kirchhoff integral]{Kirchhoff integral \cite{Brandenburg2009}}
	\label{KirchhoffScheme}
\end{figure}

This fact can be turned around so it is useful, not only for knowing the field, but to replicate it.
If, in another time and place, we manage to generate a surface acoustic field identical to the measured one, the wave field inside the volume will be the same as previous one.
For example, if we want that inside $\volumeTheo$ one can hear the sound of a string quartet (located outside $\volumeTheo$) playing the Pachelbel's Canon, we have two options. On the one hand, we hire a string quartet, make them play and the problem is solved. On the other hand, we have a more interesting solution. We record the wave field on each point of $\surfaceTheo$ when the musicians play, then we build some audio reproducing system that can replicate it, and place a person inside.

How to build that system is the issue here. Kirchhoff's integral does actually provide some answers. It states that it could be done with a surface continuous distribution of an infinite number of monopole and dipole sources, called secondary sources. This means that if, at each point of $\surfaceTheo$, there was one monopole and one dipole infinitesimal sources driven by the right signals, the replication of the virtual source field would be perfect inside $\volumeTheo$, and moreover, the field would be zero outside.

Of course, Kirchhoff-Helmholtz integral can not simply be put into practice due to obvious technical reasons. It is not practical (or even possible) to build a hollow volume with tons of tiny loudspeakers on the surface and place a listener inside.
% Even if we could, it would not have much practical use apart from an impressive virtual reality immersion.
But thankfully, in a real scenario where a finite amount of real loudspeakers are used, in realizable and simple spatial distributions, with the presence of reflective objects, diffractions, where the air is not an ideal transport media for sound propagation (which is not, since it presents air damping effects)%\cite{Brandenburg2009}
, etc., we can still aim for some degree of accuracy.
A common practical case is one where loudspeakers distributed as a straight line (not a closed surface) are used to synthesize a field only in the horizontal plane, and below a certain frequency that is inversely proportional by the separation between loudspeakers (aliasing frequency). Simplifications such as that, are necessary to implement a feasible practical system. The price to pay is the limitation of the performance in terms of accuracy, bandwidth, spatial range where it works, etc. However, it still can provide good results, depending on the requirements of the system, which are usually defined by the human hearing capabilities.
% The work of engineers consist precisely on creating real systems with practical restrictions (limited budget, inaccuracies, etc.) that meet some performance requirements.

Historically, WFS theory was developed in Delft University and first presented to the public in 1989 \cite{berkhout1989acoustic}. Since then, it has come a long way of development. During the 1990s, it was mainly a topic of research. Most of the research was focused in performing high fidelity sound reproduction to create a true immersive sound experience. 
High fidelity reproduction systems have been an important topic for many decades. Our auditory system plays a major role in how we experience our environment. It is continuously locating objects in distance and direction. Even in situations where visual cues are dominant, our ears help us analyse the environment and create the feeling of immersion. All stereo techniques (two-channel stereo, quadraphony, 5.1 and 7.1 surround sound) used in cinema or theatres share the shortcoming that only the listeners located in a very limited area, usually called sweet spot, experience good spatial immersion. In general, the more precise the spatial scene is, the smaller the sweet spot becomes. But WFS is able to synthesize a replica of the sound field over the whole listening area, and that is its biggest advantage, and the main motivation for the research \cite{Brandenburg2009}.
Various cases of successful implementation proved that WFS could actually work to some degree at least on simple scenarios \cite{Start1997,Verheijen,Vogel}.

\begin{shownto}{private}
	Other not completed content:
	\begin{enumerate}
		\item The idea was to simplify and transform Kirchhoff's integral into some other expression that was closer to real cases, and to analyse what inaccuracies  develop a real system that
		\item The goal was to design simple systems that approximated the ideal case described by Kirchhoff's integral, and to analyse what inaccuracies that derive from it. Kirchhoff's integral 
		\item The goal was to analyse how much Kirchhoff's integral could be simplified before the derived inaccuracies were unacceptable
		\item The goal was to design the simplest system...
		\item The goal was to build an implementation that proved that WFS can actually work, and it was done successfully in different cases (\cite{Start1997}\cite{Verheijen}\cite{Vogel})
	\end{enumerate}
\end{shownto}

It was not until the 21st century when commercial applications were available \cite{Brandenburg2009}: in 2003, the first cinema based on WFS started daily operation in Ilmenau (Germany), the first WFS system in a sound stage was installed in 2004 in Studio City (California, US), and since 2008 a large WFS installation is at the Chinese Theatre in Los Angeles (US) (\autoref{chineseTheatre}). In living performance, it has been used to improve spatial coherence between audio and the visual part (Bregenz Festival, Austria), or to improve speech intelligibility (auditorium of the Technical University in Berlin \cite{Musicology}). Other application areas are theme parks, virtual reality, and even the music reproduction system of the car Audi Q7. Potential applications not fully implemented yet are adjustment of multi-purpose hall room acoustics and elimination of noise disturbance and unwanted echoes.

\begin{figure}[h]
	\centering
	\includegraphics[width=0.5\columnwidth]{Img/"Mann Chinese theatre 380 channel".jpg}
	\caption{Chinese Theatre WFS installation by \emph{IOSONO} (Los Angeles, US)}
	\label{chineseTheatre}
\end{figure}

Some of the implementations are really complex and use hundreds of loudspeakers, especially in big installations, but usually they are simpler and very far from the ideal scenario that Kirchhoff's integral describes. This has the drawback that performance gets deteriorated. Despite that, since quality subjective experience is the goal in immersive sound reproduction, the criteria to evaluate whether the performance is actually good or bad are psychoacoustics. If the listener's experience is good, then it is considered a good system and there's no point in aiming at a better performance that would not be appreciated by the listener.
When psychoacoustic mechanisms for perceiving source position and width and spaciousness are considered, the necessary number of loudspeakers can be reduced drastically while maintaining good spatial listening experience and reducing computational demands. The precision of the physical replication and thus the amount of data to be processed can be reduced without audible effects \cite{Musicology}. For example, localization in the horizontal plane is much better than the perception of elevation, so using just a horizontal line array of loudspeakers does not influence that much the audio-visual experience \cite{Brandenburg2009}. Another approach is directional audio coding (DirAC): signals are separated into directional and diffuse components \cite{Musicology}.

Despite all improvements over the years, there are remaining obstacles. WFS is still at a stage of research and development, and although several approaches already enable acoustic control to a certain degree, there is a long way to go.
%Even with techniques that compensate some of the obstacles, the capability of current systems is still insufficient.
That's why nowadays WFS is usually employed as a complement to the main sound reproduction system, but not as a fully working stand-alone system.

Current research and development address mainly issues related to accessibility and improvement of performance. One important topic is the creation of easy accessible software and formats to simulate, create, control, store and play WFS content. Due to the lack of standardized loudspeaker configurations, object-based source material is necessary: combination of audio tracks with dynamic metadata that describes source position, trajectories, orientation and radiation characteristics and information on reflections and reverberation. Although several formats have been proposed, no standardized wave field synthesis format has established yet.

In order to derive the driving signals of secondary sources, the pressure field has to be known. Most approaches use mathematical models to estimate the field, because the problem of how to measure (record) a complete sound field still presents lots of limitation \cite{Avni2013}.

Another topic is the application of multiactuator panels (MAPs) so installations do not harm interior decoration due to their discreet appearance and, moreover, can be arranged continuously, preventing aliasing effects.

So far research has focused on the synthesis of virtual static monopole sources and plane waves. But more advanced features are beginning to be studied. For fast moving sources, the Doppler effect is important for an authentic sound experience. Radiation characteristics of musical instruments are complex, far from behaving as a monopole. There are also attempts to include room modes, early reflections and late reverberations \cite{Ahrens2014}. Finally, the potential of psychoacoustics is regarded as very promising by many researchers because it can help to overcome current limited acoustic control and restrictions \cite{Musicology}.

\subsection{Active Noise Control}
A problem related to sound, is the cancelation of noise. Active Noise Control (ANC) refers to a group of techniques that aim at reducing the effect of acoustic noise sources by means of an array of loudspeakers that generate a sound wave that interferes destructively with the noise wave field and, thus, cancels it. It past decades it has become a growing field of research, since passive methods are not as effective in cancelling low frequency noise \cite{Lapini2016}.

ANC has had success in cases where the listening area is very small, e.g., inside a head phone, or around a listener with restricted head movement. However, for large spaces where listeners are allowed to move freely, the problem becomes much complicated. Traditional approaches require a huge number of sensors and sources distributed within the area of interest. Moreover, this high number of sensors would constitute a highly overdetermined multiple-input multiple-output system, which causes bad convergence of adaptive algorithms and very large loudspeaker driving signals \cite{Kuntz2004}. Besides, the classical ANC adaptive filtering techniques (e.g., FxLMS and extensions) work well for minimizing the mean of some distortion measure of stationary Gaussian noise, but not for short duration noise because convergence is not achieved \cite{Lapini2016}.

There have been proposals of the use of WFS to perform ANC as a solution to previous problems \cite{Lapini2016,Kuntz2004,Zanolin1999,Morcillo2015}. The use of WFS allows to control the sound field by using a distribution of sources and sensors only on the boundary of the listening space, and listeners are not restricted in their movement and no headphones or object tracking equipment is required \cite{Kuntz2004}.

\section{Objectives}

The objective of this thesis is the study, by means of simulations, of the possibilities of performing active control noise with WFS at the audio laboratory available in the Audio and Communications Signal Processing Group (GTAC) of the Institute of Telecommunications and Multimedia Applications (iTEAM) of the Universitat Politècnica de València (UPV).

The simulations are computed on the software programming platform \emph{Matlab}. Simple scenarios are contemplated at first, and then complexity is increased in order to gain insight into the case at hand so we can provide a survey of the requirements, limitations and difficulties that may arise during the implementation of a true ANC WFS system.
