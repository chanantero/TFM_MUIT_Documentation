Wave Field Synthesis (WFS) is a method that, by means of an array of loudspeakers reproducing the proper audio signals, generates the wave field that would produce a source of sound that isn't actually there. For example, in a room where one of this arrays is set up, a person situated in any point of the room could hear the voice of a person moving through the room, as if a ghost was actually talking and walking.

WFS takes advantage of a physical principle applied to wave fields (such as acoustic waves) called Kirchhoff's integral. It states that in any source-free wave propagation media delimited by a surface S, the wave field at any point in that space can be calculated if we know the value and the gradient of the wave field on the surface S. In other words, if we want to know the sound that one can hear at any point inside S, we just have to know the acoustic pressure and gradient on S.

If we want that inside S one can hear the sound of a string quartet playing the Pachelbel's Canon outside S, we have two options. On the one hand, we hire a string quartet, make them play and the problem is solved. On the other hand, we have a more interesting solution. We record the wave field on each point of S when the musicians play, we build some system that can reproduce the same values on S, and place a person inside. That system could be a hollow sphere with infinitesimal speakers on its surface.

Of course, we cannot build a surface with trillions of microscopic loudspeakers and place a person inside. Even if we could, it wouldn't have much practical use apart from an impressive virtual reality immersion. But thankfully, even with the truncation, the use of real loudspeakers, the presence of reflective objects, positioning the sources of noise inside S, etc. the final result is still pretty good. The work of engineers consist precisely on creating real systems with practical restrictions (limited budget, inaccuracies, etc.) that meet some performance requirements.

WFS was created in Delft University in 19XX.

Another problem related to sound is the cancelation of noise. 

Active Noise Control refers to 