\chapter{Field produced by infinite line source and infinite plane.}

\section{Infinite line source}

An infinite line source situated along the $z$ axes in coordinates $x = y = 0$ will produce a field that doesn't depend on $z$ or the angle $\phi$, only on the radius $r$. The field that it produces at a point situated a distance $r$ from it, the next integral must be solved.

\begin{equation}
P(r) = \begin{aligned}
&\int_{-\infty}^{+\infty} \frac{e^{-j k \sqrt{r^2 + z^2}}}{\sqrt{r^2 + z^2}} \mathrm{d}z = 
\int_{-\pi/2}^{+\pi/2} \frac{e^{-j k r/cos{\alpha}}}{r/\cos{\alpha}} \mathrm{d}z = \\
&= \left\{ z = r \tan{\alpha} \rightarrow \dv{z}{\alpha} = r\frac{1}{\cos^2{\alpha}} \right\} = \int_{-\pi/2}^{+\pi/2} \frac{e^{-j k r/cos{\alpha}}}{\cos{\alpha}} \mathrm{d}\alpha \\
&= -\pi j \mathrm{H_0^{(2)}(k r)} = -\pi \left( \mathrm{Y}_0(k r) + j\mathrm{J}_0(k r)\right)
\end{aligned}
\end{equation}

We see that the solution is a scaled version of the zero-th order of the Hankel function of second type.

In order to appreciate what is the form of this function, we can use the stationary phase method to calculate an approximation for $kr >> 1$. The approximate solution is:
\begin{equation}
P_{a}(r) = \frac{e^{-jkr}}{\sqrt{kr}}\sqrt{\frac{2\pi}{j}}
\end{equation}

So, it is a function which amplitude decreases with the inverse of the square root of the distance, and with a linear phase change.

The approximation converges very rapidly to the real solution when $kr$ increases. \autoref{figAmpError} shows the amplitude error expressed as:
\begin{equation}
	10 \log_{10} \left( \frac{\vert P_a \vert^2 - \vert P \vert^2}{\vert P \vert^2} \right)
\end{equation}
and the phase error as
\begin{equation}
	10 \log_{10} \left( \vert phase\left(\frac{P_a}{P}\right) \vert \right)
\end{equation}
where $phase$ indicates the phase of the complex number in degrees.

\begin{figure}[h]
\centering
\includegraphics[width=0.4\textwidth]{Img/statPhaseMethApproxInfiniteLineError.eps}
\caption{Error of approximation for field produced by an infinite line source}
\end{figure}

\section{Infinite plane}
Let's assume that every surface differential in the plane XY ($z = 0$) is a monopole source. The measure point $\vec{p}$ is located at the axis Z ($x = y = 0$), at a distance $z$: $\vec{p} = [0\, 0\, z_p]^T$.

The contribution of each monopole to the point $\vec{p}$ is:
\begin{equation}
f_{partial}(x, y, z_p) = \frac{e^{-j k \sqrt{x^2 + y^2 + z_p^2}}}{\sqrt{x^2 + y^2 + z_p^2}}
\end{equation}

The field in $\vec{p}$ is:
\begin{equation}
f(z_p) = \int_{-\infty}^{+\infty} \int_{-\infty}^{+\infty} \frac{e^{-j k \sqrt{x^2 + y^2 + z_p^2}}}{\sqrt{x^2 + y^2 + z_p^2}} \mathrm{d}x \mathrm{d}y
\end{equation}

This can actually be interpreted as many differential parallel line sources located one next to the other:

\begin{equation}
f(z_p) = \int_{-\infty}^{+\infty} -\pi j \mathrm{H_0^{(2)}(k \sqrt{z^2 + y^2})} \mathrm{d}y = \frac{-j 2\pi}{k} e^{-j k z}
\end{equation}

As we see, the generated field is a plane wave, and its amplitude is inversely proportional to the frequency.

