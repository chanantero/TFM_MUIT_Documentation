Wave Field Synthesis (WFS) theory was developed in the 1990s as a new sound reproduction paradigm. Unlike stereophonic techniques, that can produce a sound image similar to that of the original sources on a small area or sweet-spot, WFS aimed at the synthesis of sound wave fronts over a volume or area. WFS theory is derived from Kirchhoff-Helmholtz equation. It states that the wave field produced by a sound source (primary source) inside a free source volume can be perfectly replicated by a surface distribution of monopole and dipole sources (secondary sources) that enclose the volume. In order to do so, secondary sources must reproduce signals that are directly proportional to the surface acoustic pressure and its directional gradient. This information can be derived from the surface geometry, and position of the primary source, it's directivity, and the signal it transmits.

A simplification can be made if the mentioned surface is a plane, and the primary sources are on one side of that plane. In that case, either a distribution of monopoles or dipoles is sufficient to replicate a field on the other side of the plane (Rayleigh I and II integrals). If, in addition, the sound sources as well as the listeners are on the same plane, just a line distribution of secondary sources, either monopoles or dipoles, is required (Rayleigh 2.5D I and II integrals). A linear array of loudspeakers can actually be approximately modelled by this last theoretical scenario and, indeed, it is the most typical type of WFS implementation in commercial applications and research so far.

Of all applications, this study has been focused on Active Noise Control (ANC). It refers to the idea of using loudspeakers to create sound fields that interfere destructively with the field generated by noise sound sources. If the sound signal and location of a noise source are known, WFS would allow us to synthesize a replica of the noise field, but with opposite sign, so it will produce noise cancellation over the area that the WFS system covers.

In order to study the possibility of using WFS based ANC in a real listening room as the one in the GTAC facilities, a series of simulations was carried out. The starting point was a simplified model were loudspeakers are substituted by ideal monopole sources and free-space conditions are assumed. At first, the main limitation we have found is the necessity of implementing a prefilter for the virtual noise source signal with frequency response $\freqFilter[frequency] = \sqrt{jk/2\pi}$. The WFS available literature that don't refer specifically to ANC, don't mention that filter or understate the importance of it. Under subjective perception criteria, this filter does not have a big effect on source location, coloration or spaciousness when synthesizing virtual sources because the human auditory system is tolerant to some types of distortions. However, if the purpose is to interfere destructively with another field, accuracy is critical. Slight phase inaccuracies can completely undermine system performance.

This prefilter is usually implemented as a FIR digital filter, but the ideal response is anticausal, so it makes necessary to delay the generation of secondary source signals by a given amount of time. The higher the FIR order (and therefore precission), the longer the delay time needed. In practice, this sets a trade-off between the system performance and the distance between the loudspeaker array and the noise source.

Since the real system is located inside GTAC's listening room, non-free space conditions were tested in simulations. A box shaped room was assumed as an idealized model of the actual listening room. A Matlab tool was used to generate acoustic path responses for different wall reflection coefficients. Of course, the higher the coefficient, the poorer the result was.

The effects of truncation were also studied. It was proven that the bad performance that we systematically got at low frequencies was caused by the fact that the length of the loudspeaker array is finite. A simple scenario was used. It was formed by a finite line of secondary sources, a primary source located at an infinite distance and a centred point of measure. It was shown that there is an almost linear relation between the distance from the measure point to the secondary line, and the minimum frequency at which the performance starts to converge.

In measures, we were able to prove that, as simulations of highly reverberant rooms suggested, it was not possible to achieve high noise cancellation levels. So, due to this technical limitation, a practical demonstration of active noise cancellation based on WFS indoors was not possible to be carried out.

\section{Future research}
%http://dissertation.laerd.com/types-of-future-research-suggestion.php
During this work, multiple questions remained unanswered. The design of the prefilter is an aspect that should be studied more thoroughly because it establishes a critical constraint. Some possible alternatives were mentioned. It is especially interesting the use of a IIR filter design as proposed in \cite{FrankSchutz2015} since it would drastically reduce the required filter order, and hence, the delay of the system.

Due to the reverberant nature of the listening room, we could not perform a good demonstration of noise cancellation with WFS. Some examples of experimental measures outdoors are available in the literature, but most of them use linear arrays, none with an octagon shaped array and the dimensions we use. Experiments in an environment that resembles more to free-space (outdoors, anechoic chamber...) are interesting. They would provide new insights that can't be drawn from computer simulations. Only after understanding the difficulties that arise in such experimental conditions, would be profitable to try the system in more practical ones. Moving directly from idealized simulations to measures in a real environment, make appear too many unknowns that are complicated to analyse and understand. A step by step process would be more reliable. For example, the issue of ground reflections has been addressed as a separate problem that can be compensated with an additional filter \cite{Lapini2018}.

Regarding truncation issues, it is convenient to explore techniques other than tapering in order to overcome bad performance at low frequencies. Other geometries for the secondary source distribution may produce different artefacts, like circular arrays or arc arrays, which have received some attention in literature. Another possibility can be to use different secondary signal processing strategies for low and mid-high frequencies. For example, apart from WFS, the other most known sound field synthesis method nowadays is Near-field Compensated Higher Order Ambisonics (NFC-HOA). Altough it is an approach theoretically restricted to spherical and circular secondary source distribution geometry and narrow-band synthesis \cite{Ahrens2012}, it 
%produces less artifacts over a restricted area \cite{Ahrens2012} and 
has been analytically proved that WFS is a generalized, high-frequency/far-field approximation of NFC-HOA \cite{FrankSchutz2015}. Hence, it is not strange that at low frequencies/near-field conditions, NFC-HOA can show better accuracy than WFS.