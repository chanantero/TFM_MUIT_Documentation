\chapter{Projection of 3D objects in subjective observation plane}

Let's say there are objects in 3D that are perfectly described by the points that constitute them. Let's say we want to take a 2D image of the 3D situation, and for doing that we project the 3D points onto a plane. This subjective observation plane is perfectly described by 3 vectors:

\begin{description}
	\item[$\hat{n}$] Unit ortogormal vector to the plane.
	\item[$\vec{u}$] Vector that belongs to the plane and indicates the direction that subjectively is interpreted as \textit{up} and its magnitude is what subjectively will be considered as a unit.
	\item[$\vec{r}_o$] 3D point that establishes the reference point of the plane and that will be taken as subjective position $[0 0]^T$.
\end{description}

We can easily calculate the vector $\vec{l}$ that indicates the subjective direction \textit{left} as $\vec{l} = \vec{u} \cross \hat{n}$. In the same way, the right direction is $\vec{ri} = \hat{n} \cross \vec{u} = -\vec{l}$.

The projection of a point $\vec{r}$ onto the subjective plane is
\begin{equation}
\vec{r}_s = \left[\frac{\vec{l}}{\norm{\vec{l}}^2}, \frac{\vec{u}}{\norm{\vec{u}}^2}\right]^T (\vec{r} - \vec{r}_o)
\end{equation}

So, subjectively, the point $\vec{r}$ is viewed as the point $\vec{r}_s$.

There are special cases where we can particularize the expressions and simplify them. For example, an interesting case is when all points we are interested in have a coordinate $z = 0$, this is, they are all contained in the $XY$ plane ($\vec{r} = [x, y, 0]^T$). Then let's define an observation plane of an observer that is looking from above:
\begin{equation}
\begin{aligned}
\vec{r}_{o, 0} &= [0, 0, h]^T \\
\hat{n}_0 &= [0, 0, -1]^T \\
\vec{u}_0 &= [0, 1, 0]^T \\
\vec{l}_0 &= [-1, 0, 0]^T \\
\vec{ri}_0 &= [1, 0, 0]^T
\end{aligned}
\end{equation}

The projections points contained in the $XY$ plane onto this observation plane is actually the $x$ and $y$ components of the real points, so we are just ignoring the z component. Taking in account that $\vec{l}_0$ and $\vec{u}_0$ are orthonormal, and orthogonal to $\vec{r}_{o,0}$:
\begin{equation}
\vec{r}_{s,0} = [\vec{l}_0, \vec{u}_0]^T \vec{r} = [-x, y]^T
\end{equation}

If, from that position, we change the observation perspective by descending an angle $\theta$ in the $\phi$ direction, the new observation plane is described by:

\begin{equation}
\begin{aligned}
	\vec{r}'_o &= h[\cos\phi\sin\theta, \sin\phi\sin\theta, \cos\theta]^T \\
    \hat{n'} &= -[\cos\phi\sin\theta, \sin\phi\sin\theta, \cos\theta]^T \\
    \vec{u}' &= [-\cos\phi\cos\theta, -\sin\phi\cos\theta, \sin\theta]^T \\
    \vec{l}' &= [\sin\phi, -\cos\phi, 0]^T
\end{aligned}
\end{equation}

Taking in account that $\vec{u}'$ and $\vec{l}'$ have magnitude $1$ and that they are orthogonal to $\vec{r}_{o,0}$, the new projection is
\begin{equation}
\vec{r}'_s = [\vec{l}', \vec{u}']^T \vec{r}
\end{equation}

As the $z$ component of all points we are interested in have value $0$, we can establish that, from now on, all 3D vectors become 2D vectors with the last value removed.

A convenient mathematical relation to establish is one where the projection $\vec{r}'_s$ is equal to a linear transformation of $\vec{r}_{s,0}$:

\begin{equation}
\vec{r}'_s = [\vec{l}', \vec{u}']^T \vec{r} = \myMatrix{T}_{2x2} [\vec{l}_0, \vec{u}_0]^T \vec{r}
\end{equation}

One way to solve it is to solve the linear system 
\begin{equation}
[\vec{l}', \vec{u}']^T = \myMatrix{T}_{2x2} [\vec{l}_0, \vec{u}_0]^T
\end{equation}
This is actually pretty straightforward since projection vectors are orthonormal. The solution is:
\begin{multline}
\myMatrix{T} = [\vec{l}', \vec{u}']^T [\vec{l}_0, \vec{u}_0] = \begin{bmatrix}
-\sin\phi & -\cos\phi \\
\cos\theta \cos\phi & -\cos\theta \sin\phi
\end{bmatrix} = \begin{bmatrix}
1 & 0\\
0 & \cos\theta
\end{bmatrix} \begin{bmatrix}
-\sin\phi & -\cos\phi \\
\cos\phi & -\sin\phi
\end{bmatrix} = \\
= \left\{\alpha = \phi + \pi/2\right\} = \begin{bmatrix}
	1 & 0\\
	0 & \cos\theta
\end{bmatrix} \begin{bmatrix}
	\cos\alpha & -\sin\alpha \\
	\sin\alpha & \cos\alpha
\end{bmatrix}
\end{multline}

If we used the vector in the right direction $\vec{ri}'$, then the transformation matrix would be
\begin{equation}
\myMatrix{T} = [\vec{ri}', \vec{u}']^T [\vec{ri}_0, \vec{u}_0] =  \begin{bmatrix}
1 & 0\\
0 & \cos\theta
\end{bmatrix} \begin{bmatrix}
\cos\alpha & \sin\alpha \\
-\sin\alpha & \cos\alpha
\end{bmatrix}
\end{equation}
A way of interpreting it is that a movement from the reference observational perspective to a new one, transforms the projection of points of the $XY$ plane with two transformations, first a rotation of $\alpha$ radians in clockwise direction, and then a scaling in the vertical dimension by $\cos\theta$.

